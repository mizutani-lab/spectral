\documentclass[11pt,letterpaper]{article}

\newcommand\doctitle{Lecture 1 Notes}
\newcommand\student{Yo Mizutani}

\usepackage{common}
\usepackage{problem_box}

%===============================================================================
%    Document body
%===============================================================================
\begin{document}

\section{Matrices for Graphs}

\begin{enumerate}[label=(\alph*)]
  \item Adjacency matrix $\bm{M} \in \R^{n \times n}$: $\bm{M}(a,b) := \begin{cases}
    w(a,b) & \text{if } ab \in E,\\
    0 & \text{if } ab \notin E,
  \end{cases}$
  where $w(a,b)$ is the weight on edge $ab$ ($w(a,b)=1$ when unweighted).
  %
  Three ways to think about such matrices:

  \begin{enumerate}[label=(\arabic*)]
    \item graph encoding
    \item operator on a vector $\bm{x} \mapsto \bm{M}\bm{x}$
    \item quadratic form on a vector $\bm{x} \mapsto \bm{x}^T \bm{M} \bm{x} \in \R$
  \end{enumerate}

  \item Degree matrix: $\bm{D} := \text{diag}(\bm{M}\bm{1})$,
  where $\bm{1} \in \R^n$ is the vector of all ones.

  \item Random walk (RW) matrix: $\bm{W} := \bm{M}\bm{D}^{-1}$
  
  Probabilities of landing at each vertex after $t$ steps of the random walk, starting at vertex $a$:
  $\bm{W}^t\bm{\delta}_a$,
  where $\bm{\delta}_a(b) := \begin{cases}
    1 & \text{if }b=a,\\
    0 & \text{if }b \neq a.
  \end{cases}$

  \begin{remark}
    $\bm{W}$ is not symmetric but similar to a symmetric matrix  (see Exercise 1).
  \end{remark}

  \item Graph Laplacian: $\bm{L} := \bm{D} - \bm{M}$

  This is discretization of $-\Delta$, the nagation of the second derivative.
  There are many important properties:

  \begin{enumerate}[label=(\arabic*)]
    \item $\bm{x}^T\bm{L}\bm{x} = \sum_{ab \in E} w(a,b)(\bm{x}(a) - \bm{x}(b))^2$ (see Exercise 2)
    \item Function $\bm{x}$ is ``smooth'' if $\bm{x}^T\bm{L}\bm{x}$ is small
    \item Functions minimizing $\bm{x}^T\bm{L}\bm{x}$ are bottom eigenfunctions of $\bm{L}$

    \textcolor{red}{What does this mean?}

    \item $\bm{L}$ is a real-valued symmetric matrix, so positive semi-definite
    \item $\lambda_1=0$
  \end{enumerate}
  
  \item Total variation: $TV_G(\bm{x}) := \sum_{ab \in E} w(a,b) \left|\bm{x}(a) - \bm{x}(b)\right|$.
  \item Laplacian for random walk: $\bm{L}_{RW} := (\bm{D}-\bm{M})\bm{D}^{-1} = \bm{I} - \bm{M}\bm{D}$
  \item Symmetric Laplacian: $\bm{L}_{SYM} := \bm{I} - \bm{D}^{-1/2}\bm{M}\bm{D}^{-1/2}$

  \textcolor{red}{Why is this useful?}

  \item Incidence matrix $\bm{B} \in \{0,1\}^{n \times m}$: $\bm{B}(v,e) := \begin{cases}
    1 &\text{if } v \in e,\\
    0 &\text{if } v \notin e.
  \end{cases}$

  Properties:
  \begin{enumerate}[label=(\arabic*)]
  \item $\bm{B}\bm{B}^T = \bm{D} + \bm{M}$.
  \item $\bm{B}^T \bm{B}=2\bm{I} + \bm{M}_{L(G)}$, where $L(G)$ denotes the line graph of graph $G$.
  \end{enumerate}
\end{enumerate}

\textbf{Observation.} Minimizing $\bm{x}^T\bm{L}\bm{x}$ is analogous to minimizing $|\nabla \bm{x}|^2$,
while minimizing $TV_G(\bm{x})$ is analogous to minimizing $|\nabla x|$.

\section{Applications}

\begin{enumerate}[label=(\alph*)]
  \item $\lambda_i$ is small $\Longrightarrow$ $\psi_i$ has low frequency;
  $\lambda_i$ is large $\Longrightarrow$ $\psi_i$ has high frequency.

  \item Graph embedding $V \to \R^{k-1}$: $i \mapsto (\psi_2(i), \psi_3(i), \ldots, \psi_k(i))$.
  \item Graph clustering
  \item Graph signal processing: think of $\psi_i$ as Fourier modes.
  \begin{enumerate}[label=(\arabic*)]
    \item extend Fourier analysis to graphical context
    \item ``convolution'' on graphs
    \item graph CNNs
    
    \textcolor{red}{Review Fourier analysis}
  \end{enumerate}

\end{enumerate}

\section{Questions}

Q. Is it easy to compute eigenvalues on large graphs?

A. It depends on sparsity, rank, etc. of the adjacency matrix.
%
Practical enough to compute a few small eigenvalues (there are numerical techniques).

\section{Exercises}

\begin{exercise} Show that $\bm{W}$ is similar to a symmetric matrix.
\end{exercise}

\begin{exercise} Show $\bm{x}^T\bm{L}\bm{x} = \sum_{ab \in E} w(a,b)(\bm{x}(a) - \bm{x}(b))^2$.
\end{exercise}

\begin{exercise}[taken from Matthew Hirn's Spring 2021 CMSE 890-001 Homework 01]
  Let $\bm{M}$ be a symmetric, real valued matrix, and let $\bm{\psi}$ and $\bm{\phi}$ be two eigenvectors of $\bm{M}$
  with eigenvalues $\mu$ and $\nu$, respectively, i.e.,

  $$\bm{M}\bm{\psi} = \mu \bm{\psi} \text{ and } \bm{M}\bm{\phi} = \nu \bm{\phi}.$$

  Prove that if $\mu \neq \nu$, then $\bm{\psi}$ must be orthogonal to $\bm{\phi}$.
  %
  Note that your proof should exploit the symmetry of $\bm{M}$, as this statement is false otherwise.
\end{exercise}

\begin{exercise}[taken from Matthew Hirn's Spring 2021 CMSE 890-001 Homework 01]
  Let $\bm{M}$ be a symmetric, real valued matrix with eigenvalues $\lambda_1,\ldots,\lambda_n$ and
  orthonormal eigenvectors $\bm{\psi}_1,\ldots,\bm{\psi}_n$.
  Let $\bm{\Psi}$ be the orthogonal matrix whose $i$-th column is $\bm{\psi}_i$.
  %
  Prove that $\bm{\Phi}^T \bm{M} \bm{\Phi} = \Lambda$,
  where $\bm{\Lambda}$ is the diagonal matrix with $\lambda_1,\ldots,\lambda_n$ on its diagonal.
  %
  Conclude that $\bm{M}=\bm{\Phi} \bm{\Lambda} \bm{\Phi}^T = \sum_{i=1}^n \lambda_i \bm{\psi}_i\bm{\psi}_i^T$.
\end{exercise}
\end{document}
